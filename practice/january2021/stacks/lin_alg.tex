\documentclass{article}
\usepackage[table]{xcolor}
\usepackage[utf8]{inputenc}
\usepackage{amssymb}
\usepackage{amsfonts}
\usepackage{amsmath}
\usepackage{multicol}

\title{Linear Algebra Review}
\author{Daniel Suh}
\date{December 31, 2021}

\begin{document}

\maketitle

\section{Matrices and Vectors}
A matrix is a rectanglur array of numbers written between square brackets. The dimensions of matrix are written as number of rows $\times$ number of columns.

\[
    \underset{\text{4 x 2 Matrix}}{
        \left[\begin{array}[]{cc}
            1402 & 191 \\
            1371 & 821 \\
            949 & 1438 \\
            147 & 1448 \\
        \end{array}\right]
    }
\]
\[
    \underset{\text{2 x 3 Matrix}}{
        \left[\begin{array}[]{ccc}
            949 & 1438 & 24\\
            147 & 1448 & 2347\\
        \end{array}\right]
    }
\]\\
A matrix can also be defined using the $\mathbb{R}$ sign:
\[
\underset{\text{This is a }\mathbb{R}^{2 \times 3} \text{ matrix.}}{
    \left[\begin{array}[]{ccc}
        1 & 2 & 3\\
        4 & 5 & 6\\
    \end{array}\right]
}
\]
\[
\underset{\text{This is a }\mathbb{R}^{3 \times 1} \text{ matrix.}}{
    \left[\begin{array}[]{c}
        1\\
        4\\
        5\\
    \end{array}\right]
}
\]\\
How do we refer to elements or entries of the matrix?
\[
    \left[\begin{array}[]{cc}
        1402 & 191 \\
        1371 & 821 \\
        949 & 1438 \\
        147 & 1448 \\
    \end{array}\right]
\]
\[
    A_{ij} = \text{$i$, $j$ entry in the $ith$ row, $jth$ column}
\]
\[
    A_{12} = 191, 
    A_{11} = 1402,
    A_{32} = 1438,
    A_{43} = undefined
\]\\
What is a vector? A vector is an $N \times 1$ matrix

\[
y = \underset{\mathbb{R}^4} {
    \left[\begin{array}[]{ccc}
        1 & 2 & 3\\
        4 & 5 & 6\\
    \end{array}\right]
}
\]
\[
    y_i = ith \text{ element}
\]
1-indexed vs 0-indexed:
\begin{align*}
    y = 
    \left[\begin{array}[]{c}
        y_1\\
        y_2\\
        y_3\\
        y_4
    \end{array}\right]
    \quad
    \quad
    \quad
    y = 
    \left[\begin{array}[]{c}
        y_0\\
        y_1\\
        y_2\\
        y_3
    \end{array}\right]
\end{align*}

Formatting for matrices and vectors:
\begin{align*}
    \text{Matrices: } A, B, C, X
    \quad \quad
    \text{Vectors: } \vec{a}, b, c, x
\end{align*}

\section{Matrix operations:}
\subsection{Matrix Addition}
Matrix addition is just simple addition:
\[
\left[\begin{array}[]{cc}
    1 & 0 \\
    2 & 0.5 \\
    1 & 3 \\
\end{array}\right]
+
\left[\begin{array}[]{cc}
    5 & -1 \\
    4 & 2 \\
    3 & 5 \\
\end{array}\right]
=
\left[\begin{array}[]{cc}
    6 & -1 \\
    6 & 2.5 \\
    4 & 8 \\
\end{array}\right]
\]
\[
\text{To add two matrices together, they must be equal dimensions}
\]
\subsection{Scalar Multiplication and Division}
Scalar means some real number. Multiply each number by this scalar
\[
3
*
\left[\begin{array}[]{cc}
    5 & -1 \\
    4 & 2 \\
    3 & 5 \\
\end{array}\right]
=
\left[\begin{array}[]{cc}
    15 & -3 \\
    12 & 6 \\
    9 & 15 \\
\end{array}\right]
\]
\[
\text{The dimensions will be the same from the beginning to the end:}
\]
\[
\frac{
\left[\begin{array}[]{cc}
    5 & -1 \\
    4 & 2 \\
    3 & 5 \\
\end{array}\right]
}{4}
=
\frac{1}{4}
\times 
\left[\begin{array}[]{cc}
    5 & -1 \\
    4 & 2 \\
    3 & 5 \\
\end{array}\right]
\]

\subsection{Multiple Operations}
\[
3
*
\left[\begin{array}[]{c}
    1\\
    4\\
    2
\end{array}\right]
+
\left[\begin{array}[]{c}
    0\\
    0\\
    5
\end{array}\right]
-
\frac{
\left[\begin{array}[]{c}
    3\\
    0\\
    2
\end{array}\right]
}{3}
=
\left[\begin{array}[]{c}
    3\\
    12\\
    6
\end{array}\right]
+
\left[\begin{array}[]{c}
    0\\
    0\\
    5
\end{array}\right]
-
\left[\begin{array}[]{c}
    1\\
    0\\
    \frac{2}{3}
\end{array}\right]
=
\left[\begin{array}[]{c}
    2\\
    12\\
    \frac{31}{3}
\end{array}\right]
\]
\subsection{Matrix-Vector Multiplication}
\[
\left[\begin{array}[]{cc}
    1 & 3\\
    4 & 0\\
    2 & 1
\end{array}\right]
\times
\left[\begin{array}[]{c}
    1\\
    5
\end{array}\right]
=
\left[\begin{array}[]{c}
    1 \times 1 + 3 \times 5 = 16\\
    4 \times 1 + 0 \times 5 = 4\\
    2 \times 1 + 1 \times 5 = 7
\end{array}\right]
\]
A $3 \times 2$ matrix multiplied by a $2 \times 1$ matrix results in a $3 \times 1$ matrix. \\\\
To get the vector result, you multiply A's $ith$ row with elements of vector $x$ and add them up.
\[
    \underset{\text{M x N Matrix}}{
        \left[\begin{array}[]{cc}
            ...
        \end{array}\right]
    }
    \times
    \underset{\text{N x 1 Matrix}}{
        \left[\begin{array}[]{cc}
            ...
        \end{array}\right]
    }
    =
    \underset{\text{M-dimensional vector}}{
        \left[\begin{array}[]{cc}
            ...
        \end{array}\right]
    }
\]
Visually, this is how to multiply them together:
\[
\left[\begin{array}[]{cc}
    \rowcolor{red!20}
    1 & 3\\
    4 & 0\\
    2 & 1
\end{array}\right]
\times
\left[\begin{array}[]{>{\columncolor{blue!20}}c}
    1\\
    5
\end{array}\right]
=
\left[\begin{array}[]{c}
    \rowcolor{yellow!20}
    1 \times 1 + 3 \times 5 = 16\\
    4 \times 1 + 0 \times 5 = 4\\
    2 \times 1 + 1 \times 5 = 7
\end{array}\right]
\]
\subsection{Hypothesis Function with Vectors}
\[
    h_{\theta}(x) = -40 + 0.25x
\]
\begin{center}
    \begin{tabular}{ |c| }
        \hline
        Housing Prices:\\
        2104\\
        1416\\
        1534\\
        852\\
        \hline
    \end{tabular}\\
\end{center}
Now, creating a matrix out of these values:
\[
\left[\begin{array}[]{cc}
    1 & 2104\\
    2 & 1416\\
    1 & 1534\\
    1 & 852
\end{array}\right]
\times
\left[\begin{array}[]{c}
    -40\\
    0.25
\end{array}\right]
=
\underset{\text{4 x 1 Matrix}}{
    \left[\begin{array}[]{c}
        -40\times1+0.25\times2104 = h_{\theta}(2104)\\
        -40\times1+0.25\times1416 = h_{\theta}(1416)\\
        ...
    \end{array}\right]
}
\]
\subsection{Matrix-Matrix Multiplication}
\[
\left[\begin{array}[]{ccc}
    1 & 3 & 2\\
    4 & 0 & 1
\end{array}\right]
\times
\left[\begin{array}[]{cc}
    1 & 3\\
    0 & 1\\
    5 & 2
\end{array}\right]
=
\left[\begin{array}[]{cc}
    1 \times 1 + 3 \times 0 + 2 \times 5 & 1 \times 3 + 3 \times 1 + 2 \times 2\\
    ... & ...\\
\end{array}\right]
\]
A $2 \times 3$ matrix multiplied by a $3 \times 2$ matrix results in a $2 \times 2$ matrix. \\\\
To multiply a matrix by a matrix, the $N$ must match in both matrices:\\
\[
    \underset{\text{M x N Matrix}}{
        \left[\begin{array}[]{cc}
            ...
        \end{array}\right]
    }
    \times
    \underset{\text{N x O Matrix}}{
        \left[\begin{array}[]{cc}
            ...
        \end{array}\right]
    }
    =
    \underset{\text{M x O vector}}{
        \left[\begin{array}[]{cc}
            ...
        \end{array}\right]
    }
\]
Visually, this is how to multiply them together:
\[
\left[\begin{array}[]{ccc}
    \rowcolor{red!20}
    1 & 3 & 5\\
    \rowcolor{blue!20}
    4 & 0 & 5
\end{array}\right]
\times
\left[\begin{array}[]{>{\columncolor{red!20}}c>{\columncolor{blue!20}}c}
    1 & 3\\
    5 & 3\\
    1 & 1
\end{array}\right]
=
\left[\begin{array}[]{c}
    \rowcolor{red!20}
    ...\\
    \rowcolor{blue!20}
    ...
\end{array}\right]
\]
The $ith$ column of the matrix $C$ is obtained by multiplying $A$ with the $ith$ column of $B$. (for $i=1, 2, ..., O$)

\subsubsection{Competing Hypotheses}
\[
    h_{\theta}(x) = -40 + 0.25x
\]
\[
    h_{\theta}(x) = 200 + 0.1x
\]
\[
    h_{\theta}(x) = -150 + 0.4x
\]
\begin{center}
    \begin{tabular}{ |c| }
        \hline
        Housing Prices:\\
        2104\\
        1416\\
        1534\\
        852\\
        \hline
    \end{tabular}\\
\end{center}
Now, creating a matrix out of these values:
\[
\left[\begin{array}[]{cc}
    1 & 2104\\
    2 & 1416\\
    1 & 1534\\
    1 & 852
\end{array}\right]
\times
\left[\begin{array}[]{ccc}
    -40 & 200 & -150\\
    0.25 & 0.1 & 0.4
\end{array}\right]
=
\left[\begin{array}[]{>{\columncolor{red!20}}c>{\columncolor{blue!20}}c>{\columncolor{yellow!20}}c}
    486 & 410 & 692\\
    314 & 342 & 416\\
    344 & 353 & 464\\
    173 & 285 & 191
\end{array}\right]
\]
With this, you can very quickly make multiple predictions

\subsection{Matrix Multiplication Properties}
\subsubsection{Communative}
Scalar by a matrix is communative
\[
    3
    *
    \left[\begin{array}[]{ccc}
        x_1 & x_2 & x_3 \\
        x_4 & x_5 & x_6 
    \end{array}\right]
    =
    \left[\begin{array}[]{ccc}
        x_1 & x_2 & x_3 \\
        x_4 & x_5 & x_6 
    \end{array}\right]
    *
    3
\]
However, this is not true with a matrix. Let $A$ and $B$ be matrices. Then in general, $A \times B \neq B \times A$.
\subsubsection{Assocative}
\[
    3 \times 5 \times 2 = 15 \times 2 = 10 \times 3
\]
This ends up being true for matrices, so $A \times B \times C = A \times (B \times C) = (A \times B) \times C$

\subsubsection{Identity Matrix}
1 is considered identity. This is because $1 \times z = z \times 1 = z$ for any z. An identity matrix is denoted with a $I_{n \times n}$
\[
    \left[\begin{array}[]{ccc}
        1 & 0 & 0 \\
        0 & 1 & 0 \\
        0 & 0 & 1
    \end{array}\right]
\]
\[
    \left[\begin{array}[]{cccc}
        1 & 0 & 0 & 0\\
        0 & 1 & 0 & 0\\
        0 & 0 & 1 & 0\\
        0 & 0 & 0 & 1\\
    \end{array}\right]
\]
For any matrix $A$,
\[
    A \times I = I \times A = A
\]
Although in most matrices, $AB \neq BA$, if $B$ is an identity matrix, then this is true. 

\subsection{Matrix Inverse and Transpose}
\subsubsection{Matrix Inverse}
Each number in most $\mathbb{R}$ has an inverse that gives back $1$:
\[
    3 \times (3^{-1}) = 1
\]
\[
    12 \times (12^{-1}) = 1
\]


\end{document}
